\documentclass[12pt]{article}
\usepackage{mathtools,amssymb,amsthm}
\usepackage{pgfplots,fullpage}
\pgfplotsset{compat=1.15}
\usepackage{enumitem}
% \newcommand{\sol}{\textbf{Solution.}}
\newcommand{\sol}{\textbf{\textit{Solution: }}}

\begin{document}
\begin{center} \large
\textbf{Stellenbosch Camp December 2024}

\textbf{Intermediate Section Test 3 Solutions}

\textbf{Time: \(2 \frac{1}{2}\) hours}
\end{center}

\begin{enumerate}[topsep=2\bigskipamount,itemsep=\bigskipamount]
\item \textit{Find all pairs $(a,b)$ of real numbers with $a+b = 1$, which satisfy $(a^2+b^2)(a^3+b^3)=a^4+b^4$.}

\sol Since \(a + b = 1\), we have \(a^{3} + b^{3} = (a + b)(a^{2} - ab + b^{2}) = a^{2} - ab + b^{2}\). Therefore,
\[\begin{aligned}
& & (a^{2} + b^{2})(a^{3} + b^{3}) & = a^{4} + b^{4}\\
& \iff & (a^{2} + b^{2})(a^{2} - ab + b^{2}) & = a^{4} + b^{4}\\
& \iff & a^{4} - a^{3}b + 2a^{2}b^{2} - ab^{3} + b^{4} & = a^{4} + b^{4}\\
& \iff & 0 & = a^{3}b - 2a^{2}b^{2} + ab^{3} = ab(a - b)^{2}.
\end{aligned}\]
The last statement is true if and only if \(a = 0\) or \(b = 0\) or \(a - b = 0\). Therefore, the solutions \((a, b)\) are \((0, 1)\), \((1, 0)\), and \((\frac{1}{2}, \frac{1}{2})\).

\item \textit{For which values of $n$ is $1! + 2! + 3! + \dots + n!$ a perfect square?}

\sol Notice that for $n\geqslant5$ we have that $10\mid n!$. Therefore for all $n\geqslant 4$ we have that $1!+2!+3!+\dots+n!\equiv 1!+2!+3!+4!\equiv 3 \pmod{10}$\\ But, $3$ is a quadratic non-residue modulo $10$.\\
Therefore $n<4$. We can now check the remaining cases:
\begin{itemize}
    \item $n=1, 1!=1$, which is a square.
    \item $n=2, 1!+2!=3$, which is not a square.
    \item $n=3, 1!+2!+3!=9$, which is a square.
\end{itemize}
Therefore $n=1$ and $n=3$ are the only solutions.
(In other words for $n\geqslant 5$, $n!$ ends in a zero. Therefore for $n\geqslant4$ the sum of $1!+2!+3!+\dots+n!$ ends in the same digit as $1!+2!+3!+4!$ ends in, which is a $3$, but no squares end in a $3$.)

\item \textit{Hugo and Jared play the following game. They take turns writing the numbers 1 to 72 on a $8 \times 9$ grid of squares. The numbers may be used in any order, but each number can only be used once. Hugo goes first. At the end of the game, Hugo gets a point for every row or column with an odd sum, and Jared gets a point for every row or column with an even sum. The winner is the one with the higher score. Does one of the players always have a winning? If so, who is it and how?}

\sol WLOG let there be 9 columns and 8 rows. Jared has a winning strategy. Whenever Hugo makes a play, Jared plays a number of the opposite parity in the same column (he will always be able to do this since both the positions and the numbers can be arranged in pairs). This will cause all the columns to have an even sum, so Jared will get at least 9 points out of a possible 17, thus he wins.
Notice Jared can also win by always playing a number of the same parity in the same way.

\item \textit{The diagonals $AC$ and $BD$ of a cyclic quadrilateral $ABCD$ intersect at $P$. The circumcircle of the triangle $PDC$ intersects $BC$ at a point $E$, beyond $C$, and intersects $AD$ at a point $F$, beyond $D$. The circumcircle of the triangle $PAB$ intersects $BC$ at a point $H$, between $B$ and $C$, and intersects $AD$ at a point $G$, beyond $A$. Prove that $E$, $F$, $G$, and $H$ lie on the circumference of a circle with centre $P$.}

\definecolor{uuuuuu}{rgb}{0.26666666666666666,0.26666666666666666,0.26666666666666666}
\definecolor{uququq}{rgb}{0.25098039215686274,0.25098039215686274,0.25098039215686274}
\begin{center}
\begin{tikzpicture}[line cap=round,line join=round,x=1cm,y=1cm, scale=0.8]
\tikzstyle{every node} = [font=\large]
% \clip(-9.36,-5.73) rectangle (15.36,9.81);
\draw [line width=2pt] (-2.9631577298469693,1.0085340402458127) circle (4.525182915383754cm);
\draw [line width=2pt] (-0.36,4.71)-- (-0.3,-2.65);
\draw [line width=2pt] (-7.08,-0.87)-- (1.4291974663749007,2.096879687503036);
\draw [line width=2pt] (1.3705004261601699,-0.5701347267825656) circle (2.667660253581243cm);
\draw [line width=2pt] (-4.667344728080366,3.060888274677429) circle (4.612243281649191cm);
\draw [line width=2pt] (-4.4967479140850255,-1.5481989252107162)-- (1.8039052492878591,-3.202352705565249);
\draw [line width=2pt] (-7.08,-0.87)-- (-0.3,-2.65);
\draw [line width=2pt] (-1.572119682439728,6.480299043502523)-- (3.8782271825381485,-1.4799246717056354);
\draw [line width=2pt] (-0.36,4.71)-- (1.4291974663749007,2.096879687503036);
\draw [line width=2pt] (1.4291974663749007,2.096879687503036)-- (-0.3,-2.65);
\draw [line width=2pt] (-0.36,4.71)-- (-7.08,-0.87);
\draw [line width=2pt] (-1.572119682439728,6.480299043502523)-- (-4.4967479140850255,-1.5481989252107162);
\draw [line width=2pt] (3.8782271825381485,-1.4799246717056354)-- (1.8039052492878591,-3.202352705565249);
\draw [line width=2pt] (-4.4967479140850255,-1.5481989252107162)-- (3.8782271825381485,-1.4799246717056354);
\draw [line width=2pt] (-0.333686572073226,1.4822195076490514)-- (3.8782271825381485,-1.4799246717056354);
\draw [line width=2pt] (1.8004030088102878,0.12806688452196682) -- (1.6623405554585429,-0.06824602634865755);
\draw [line width=2pt] (1.8822000550063809,0.07054086229207428) -- (1.744137601654636,-0.1257720485785501);
\draw [line width=2pt] (-4.4967479140850255,-1.5481989252107162)-- (-0.333686572073226,1.4822195076490514);
\draw [line width=2pt] (-2.5262637220812487,0.03460229889190373) -- (-2.3850190769221764,-0.15943365193651401);
\draw [line width=2pt] (-2.445415409236075,0.0934542343748501) -- (-2.304170764077003,-0.10058171645356766);
\draw [line width=2pt] (-1.572119682439728,6.480299043502523)-- (-0.333686572073226,1.4822195076490514);
\draw [line width=2pt] (-0.8484509205673498,4.058652663651076) -- (-1.0814061992389656,4.000930586947004);
\draw [line width=2pt] (-0.8244000552739885,3.9615879642045706) -- (-1.0573553339456043,3.903865887500499);
\draw [line width=2pt,dash pattern=on 3pt off 3pt] (-0.33368657207322655,1.4822195076490496) circle (5.14922475869614cm);
\begin{scriptsize}
\draw [fill=uququq] (-0.36,4.71) circle (2pt);
\draw[color=uququq] (0,5.1) node {$A$};
\draw [fill=uququq] (-7.08,-0.87) circle (2pt);
\draw[color=uququq] (-6.92,-0.3) node {$B$};
\draw [fill=uququq] (-0.3,-2.65) circle (2pt);
\draw[color=uququq] (-0.14,-3.2) node {$C$};
\draw [fill=uququq] (1.4291974663749007,2.096879687503036) circle (2pt);
\draw[color=uququq] (1.65,2.48) node {$D$};
\draw [fill=uuuuuu] (-0.333686572073226,1.4822195076490514) circle (2pt);
\draw[color=uuuuuu] (-1,1.88) node {$P$};
\draw [fill=uuuuuu] (1.8039052492878591,-3.202352705565249) circle (2pt);
\draw[color=uuuuuu] (1.9,-2.6) node {$E$};
\draw [fill=uuuuuu] (3.8782271825381485,-1.4799246717056354) circle (2pt);
\draw[color=uuuuuu] (4.5,-1.5) node {$F$};
\draw [fill=uuuuuu] (-4.4967479140850255,-1.5481989252107162) circle (2pt);
\draw[color=uuuuuu] (-4.7,-2) node {$H$};
\draw [fill=uuuuuu] (-1.572119682439728,6.480299043502523) circle (2pt);
\draw[color=uuuuuu] (-1.42,6.88) node {$G$};
\end{scriptsize}
\end{tikzpicture}
\end{center}

\sol Since $BHPAG, ABCD, PDFEC$ are all cyclic, we have the following results: 
\[
\angle HGF\equiv\angle HGA = \angle HBA\equiv\angle CBA = \angle CDF =  180^{\circ}-\angle HEF
\] showing $EFGH$ cyclic. Then 
\[
\angle PFD = \angle PCD\equiv \angle ACD=\angle ABD\equiv ABP = \angle AGP
\] and thus $PF=PG$. Then 
\[
\angle HGP=\angle HBP\equiv\angle CBD=\angle CAD=\angle GHP
\] and thus $PG=PH$. Since $PF=PG=PH$ we have $P$ is the centre of circle $(FGH)$ but since $EFGH$ is cyclic the result follows. $\hfill\blacksquare$

\item \textit{Find all pairs \((p, q)\) of prime numbers such that, for some positive integer \(a\), we have
\[p^{2} + 5pq + 4q^{2} = a^{2}.\]}

\sol Suppose \((p, q)\) and \(a\) are as in the question. We have
\[a^{2} - pq = p^{2} + 4pq + 4q^{2} = (p + 2q)^{2},\]
so
\[pq = a^{2} - (p + 2q)^{2} = (a - p - 2q)(a + p + 2q).\]
Since \(pq\) is positive and \(a + p + 2q\) is positive, so is \(a - p - 2q\). Now \(a + p + 2q > p\) and \(a + p + 2q > q\), but \(p\) and \(q\) are prime, so \(a + p + 2q = pq\) and \(a - p - 2q = 1\), so \(a = p + 2q + 1\) and \((p + 2q + 1) + p + 2q = pq\), so \(2p + 4q + 1 = pq\), so
\[9 = pq - 2p - 4q + 8 = (p - 4)(q - 2).\]
Therefore, \((p - 4, q - 2)\) is among
\[(-9, -1), (-3, -3), (-1, -9), (1, 9), (3, 3), (9, 1).\]
It follows that \((p, q)\) is among
\[(-5, 1), (1, -1), (3, -7), (5, 11), (7, 5), (13, 3).\]
Since \(p\) and \(q\) are positive and prime, we find that
\[(p, q) = (5, 11) \textrm{ or } (p, q) = (7, 5) \textrm{ or } (p, q) = (13, 3).\]
Those three pairs are the solutions, because
\[\begin{aligned}
5^{2} + 5(5)(11) + 4(11)^{2} & = 784 = 28^{2},\\
7^{2} + 5(7)(5) + 4(5)^{2} & = 324 = 18^{2}\textrm{, and}\\
13^{2} + 5(13)(3) + 4(3)^{2} & = 400 = 20^{2}.
\end{aligned}\]
This completes the proof. $\hfill\blacksquare$
\end{enumerate}
\end{document}