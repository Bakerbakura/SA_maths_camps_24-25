\documentclass[12pt]{article}
\usepackage{mathtools,amssymb}
\usepackage{graphicx,enumitem,fullpage,tikz}
\usepackage{multicol}

\title{Beginner Test 3 - Solution}
\author{Stellenbosch Camp 2024}
\date{Friday 13 December 2024}

\begin{document} \maketitle

Note that the below problems are \emph{not} arranged in order of difficulty.

\begin{enumerate}[topsep=2\bigskipamount,itemsep=\bigskipamount]

\item The closest time to when the 4 digits will again appear together is ${\bf 04:26}$. That is in $1$ hour $40$ minutes or in $100$ minutes.

\item We have $ab = 10a + b, ba = 10b+a,$ and $ca6=100c+10a+6$. So $ab+ba=ca6$ implies that $10a+b + 10b+a=100c+10a+6$ or $11b =100c-a+6.$ So $100c-a+6$ must be divisible by $11$. Taking $c=1$, we have $11b=106-a$, which in turn means that $106-a$ must be divisible by $11.$ Taking $a=7,$ $106-7=99$ is divisible by $11$. Now $11b=106-7=99$ implies that $b=9$. Hence, $a+b+c=1+9+7=17$.

\item The last digit of the number $17^{100}$ is the remainder when $17^{100}$ is divided by $10$.
But 
\begin{align*}
    17 &\equiv 7 \pmod{10}, & 17^2 &\equiv 7^2\equiv 9 \pmod{10}, \\
    17^3 &\equiv 3 \pmod{10}, & \text{and}\quad 17^4 & \equiv 1 \pmod{10}.
\end{align*}

Performing the Euclidean division of $100$ by $4$ gives  $100=4\cdot 25$.

So $ 17^{100} = 17^{4\cdot 25} = (17^{4})^{25}$ and  $17^4 \equiv 1 \pmod{10}$ implies that $(17^4)^{25} \equiv 1^{25} \equiv 1 \pmod{10}$.
Hence the remainder when $17^{100}$ is divided by $10$ (or the last digit of $17^{100}$) is $1$.

\item Determine the remainder when $38^{24}+39^{77}$ is divided by $7$\\
We have 
\[ 38 \equiv 3 \ (mod \ 7), \; 38^2 \equiv 3^2 \equiv 2\ (mod \ 7),\; 38^3\equiv 6\ (mod \ 7), \; 38^4\equiv 4\ (mod\ 7),\; 38^5\equiv 5\ (mod \ 7).\] 
Lastly, $38^6 \equiv 1\ (mod \ 7).$ Further $24 = 6\cdot 4$, so $38^{24} = (38^{6})^{4} \equiv 1\ (mod \ 7)$ since $38^6\equiv 1\ (mod\ 7)$.  So $1$ is the remainder when $38^{24}$ is divided by $7.$ On the other hand,

\[ 39\equiv 4\ (mod\ 7),\; 39^2\equiv 2\ (mod\ 7),\; 39^3\equiv 1\ (mod \ 7)\]

and $77 = 3\cdot 25 + 2$. So $39^{77}=(39^3)^{25}\cdot 39^2$, which implies that
$39^{77}\equiv  (39^3)^{25}\cdot 39^2 \ (mod\ 7)$ or $39^{77}\equiv  39^2 \ (mod\ 7)$ or $39^{77}\equiv  2 \ (mod\ 7)$. So $2$ is the remainder when $39^{77}$ is divided by $7$. In conclusion, the remainder when $38^{24}+39^{77}$ is divided by $7$ is $1+2=3.$

 \item 

 \item $(x^{x^4})^4 = 64^4.$ So $(x^4)^{x^4} = (8^2)^4=8^8.$ Therefore $x^4=8$ or $x =\pm\sqrt[4]{8}.$
 
 \item 
 
 \item If $n^{2}-440$ is a perfect square, then $n^{2}-440=m^{2}$ for some integer $m$. We can assume that $m\geq 0$.We then have 
\begin{equation*}
    n^{2}-m^{2}=440.
\end{equation*}
Note that $440 = 8\cdot5\cdot11$. $n^{2}-m^{2}$ is the difference of two squares, so 
we can write 
\begin{equation*}
    (n-m)(n+m)= n^{2}-m^{2}=8\cdot5\cdot11.
\end{equation*}
Since $n^{2}-m^{2}$ is even, both $n,m$ are even or odd. Then $n+m$ and $n-m$ are both even. Since $(n+m)(n-m)=8\cdot5\cdot11$, $4$ must be a factor of $n+m$ or $n-m$ and $2$ is a factor of the remaining one. The factors $5,11$ can be accounted for in four different ways.
\begin{equation*} \{n+m,n-m\}= 
    \begin{array}{cc}
 &   \{4\cdot 5\cdot 11,2\}\\
     & \{4\cdot5, 2\cdot11 \}\\
     &\{4\cdot11, 2\cdot5\}\\
     & \{2\cdot5\cdot11, 4\}
\end{array}.
\end{equation*}
Since $m\geq0$ by assumption, $n-m$ has to be smaller than $n+m$. Solving for $n,m$ above, we find 4 integer solutions 
\begin{equation*}
    (n,m)= (111,109),\;(21,1),\;(27,17),\; (57,53).
\end{equation*}

\item Taking the $100$-th root on both sides gives 
\begin{equation*}
    \pm (n^{2})<5^{3}=125.
\end{equation*}
The largest perfect square smaller than $125$ is $121$. This means that $-11,11$ are solutions to the inequality. Since $11$ is the largest integer, the answer is $11$.

\item Let $z=x+1$. Then $x=z-1$. We then write the function in terms of $z$.
\begin{equation*}
    f(z)= (z-1)^{2}+5(z-1)+3.
\end{equation*}
Note that $x-1=z-1-1=z-2$. So $f(x-1)=f(z-2)$ which is 
\begin{align*}
    f(x-1)=f(z-2)&=(z-2-1)^{2}+5(z-2-1)+3\\
    &=(z-3)^{2}+5(z-3)+3\\
    &=(x-2)^{2}+5(x-2)+3\\
    &= x^{2}-4x+4+5x-10+3\\
    &= x^{2}+x-3.
\end{align*}

\item

\item 

\item 

 \item By drawing the circles in and drawing a grid of squares of side length 4 around the circles, we notice that the star-shaped areas between circles are comprised of 4 areas between the corner of the squares and the circles inside them. $\checkmark$ This means that each star-shaped region is $16-4\pi$ units$^2$. This means that the total area of the shaded region is $2(16-4\pi)+2(4\pi) = 32$. $\checkmark \checkmark \checkmark$\textbf{[4]} 
    \item We shall prove this by assuming CE = 7 units and arrive at a contradiction. $\checkmark$ 
    Let $CE = 7\checkmark$. Let $DE = a$ and $AD = BC = b$. Then $AB = a + 7$, and so
\[
(a + 7)^2 = AB^2 = AE^2 + BE^2 = AD^2 + DE^2 + EC^2 + BC^2, \checkmark
\]
\[
= b^2 + a^2 + 7^2 + b^2.
\]

This implies:
\[
a^2+14a+49 = a^2 +2b^2+49 \checkmark
\]
\[
\implies 7a = b^2.
\]

But then $a=7x^2$ for some $x$.\checkmark
Then, in triangle $\triangle CBE$,
\[
BE^2 = 7^2 + b^2 = 7^2 + 7a = 7^2 + 7^2x^2 = 7^2(x^2+1). \checkmark
\]

But then it must be that $x^2+1 \checkmark$ is a perfect square, which is untrue for all $x>1\checkmark$. So CE cannot be 7 units long.\textbf{ [8]}

\item The upper triangle is covered by $A$, two triangles with area half of $A$, and one triangle with area a quarter of $A\checkmark$.  
The lower triangle is covered by $B$ and two triangles with area half of $B\checkmark$.  
The upper and lower triangles have the same area, so:
\[
\left(1 + 2 \left(\frac{1}{2}\right) + 1 \left(\frac{1}{4}\right)\right) \times \text{Area } A = \left(1 + 2 \left(\frac{1}{2}\right)\right) \times \text{Area } B, \checkmark
\]
\[
\implies \text{Area } A = \frac{8}{9} \times \text{Area } B.\checkmark
\]

\textbf{Answer:} $\frac{8}{9} \checkmark \checkmark$ \textbf{[6]}

\item \[
\text{Area of } DEBF = \text{Total Area} - \text{Unshaded Region} \checkmark
\]

\[
= 36 - (\text{Area of } DGE + \text{Area of } AHEG + \text{Area of } EHB + \text{Area of } BF I + \text{Area of } F ICJ + \text{Area of } DF J) \checkmark \checkmark
\]

\[
= 36 - \left(
\frac{1}{2} \cdot 12 + 12 + 
\frac{1}{2} \cdot 6 + 
\frac{1}{2} \cdot 4 + 2 + 
\frac{1}{2} \cdot 10
\right) 
\]

\[
= 36 - 30 = 6.
\]

\textbf{Answer:} $6 \, \text{cm}^2 \checkmark$ \textbf{[4]}


\end{enumerate}

\end{document}