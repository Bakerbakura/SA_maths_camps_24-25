\documentclass[12pt]{article}
\usepackage{mathtools,amssymb}
\usepackage{graphicx,enumitem,fullpage,tikz}
\usepackage{multicol}

\title{Beginner Test 2 Memorandum}
\author{Stellenbosch Camp 2024}
\date{Wednesday 10 December 2024}

\begin{document} \maketitle

\begin{enumerate}[topsep=\bigskipamount,itemsep=\bigskipamount]

\item Because the numbers are consecutive, taking the total of the $9$ included numbers and finding the average gives us a good idea of the possible middle number.
\[ 9832 : 9=1092 \frac{4}{9}. \]
But the middle number must be odd and the middle of the $11$ numbers must also be odd.

Suppose the middle number is $1091$, then the 11 numbers would range from $1086$ to $1096$. Their sum would be $1091 \times 11 = 12001$ then $12001 - 9832 = 2169$ which means the two omitted numbers would be $1084$ and $1085,$ however these are outside of the expected range.

Suppose the middle number is $1093$, then the 11 numbers would range from $1088$ to $1098$. Their sum would be $1093 \times 11 = 12023$ then $12023 - 9832 = 2191$ which means the two omitted numbers would be $1095$ and $1096$, which are within the expected range.
We could check other nearby odd numbers, such as $1095$, however this will not work. Therefore, the correct total is $12023$.

\item To be divisible by $33$, the number must be divisible by $3$ and by $11$.
For divisibility by $3$, the sum of the digits must be a multiple of $3$.
For divisibility by $11$, the difference between the sum of the odd digits and the even digits must be a multiple of $11$.

In $\overline{x2014y}$, the sum of the given digits is $7$, meaning that $x+y$ needs to be $2, 5, 8, 11, 14$ or $17$ (the maximum since $x$ and $y$ are at most $9$).
Considering the alternate digit sums, we would need $2+1+y-x-0-4 = y-x-1$ to be a multiple of $11$.
This gives us $x=8$ and $y=9$. 
Hence the number is $820149$.

\item Fig. 1 has $9$ shaded squares, Fig. 2 has $12$, and Fig. 3 has $15$. For a figure with $n$ squares on the sides, there is always $n$ shaded squares on the diagonal and it remains $n-1$ shaded squares on the left side and $n-2$ shaded squares at the bottom side. For Fig. 30, $n=33$. So there are $33+32+31 = 96$ shaded squares in total.

\item We first use the division algorithm to find the $\text{HCF}(123,36)$
\begin{eqnarray*}
    123 &=& 3\times 36 + 15\\
    36 &=& 2\times 15 + 6\\
    15 &=& 2\times 6 + 3\\
    6 &=& 2\times 3 + 0
\end{eqnarray*}
The last non-zero remainder is the $\text{HCF}$ of $123$ and $36$. So $\text{HCF}(123,36)=3$.
Now, solving backward for the remainder starting from the second last equation, we get 
\begin{align*}
    3 &= 15 - 2\times 6\\
    &= 15 - 2\times (36 - 2\times 15) = 5\times 15 - 2\times 36\\
    &= 5(123 - 3\times 36) -2\times 36 = 5\times 123 - 17\times 36.
\end{align*}
So, $5\times 123 - 17\times 36  = 3$.
Identifying this equation with the initial equation $123a + 36b=3$, we see that $a=5$ and $b=-17$ works.


\item We can expand the product as follows:
\begin{align*}
    &\mspace{20mu} \left(1-\frac{1}{2^2}\right) \left(1-\frac{1}{3^2}\right) \left(1-\frac{1}{4^2}\right) \dotsm \left(1-\frac{1}{10^2}\right) \\
    &= \left(1-\frac{1}{2}\right) \left(1+\frac{1}{2}\right) \left(1-\frac{1}{3}\right) \left(1+\frac{1}{3}\right) \left(1-\frac{1}{4}\right) \left(1+\frac{1}{4}\right) \dotsm \left(1-\frac{1}{10}\right) \left(1+\frac{1}{10}\right) \\
    &= \frac{1}{2} \times \frac{3}{2} \times \frac{2}{3} \times \frac{4}{3} \times \frac{3}{4} \times \frac{5}{4} \dotsm \frac{9}{10} \times \frac{11}{10} \\
    &= \frac{1}{2} \times \frac{11}{10} = \frac{11}{20}.
\end{align*}

\item We can write  $\dfrac{a^2}{b^2}+\dfrac{b^2}{a^2} = 7$ as $\bigg(\dfrac{a}{b}\bigg)^2 + \bigg(\dfrac{b}{a}\bigg)^2 = 7$.
But 
\begin{equation*}
 \left(\dfrac{a}{b}+  \dfrac{b}{a}\right)^2  =  \bigg(\dfrac{a}{b}\bigg)^2 + \bigg(\dfrac{b}{a}\bigg)^2  + 2\times \bigg(\dfrac{a}{b}\bigg) \bigg(\dfrac{b}{a}\bigg) = 7 + 2 =9.
\end{equation*}
So, $\dfrac{a}{b}+  \dfrac{b}{a} = \pm 3$.
\item $2023^2 -2022^2 = (2023-2022)(2023+2022) = 1 \times 4045 = 4045$.

\item By trial and error we can see that for $m=4$, $3^m-2^m = 3^4-2^4 = 81-16 = 65$.

\item \textbf{Solution 1:} By Pythagoras' Theorem we see that
\[ AC^2 = AB^2 +BC^2 = 15^2 +20^2 = 225+400 = 625 = 25^2 \implies AC = 25. \]
Now triangle $BDC$ is similar to triangle $ABC$, since they have a common angle at $C$ and $\angle ABC = \angle BDC = 90^\circ$.
Thus
\[ \frac{BD}{AB} = \frac{BC}{AC} \implies BD = \frac{AB \times BC}{AC} = \frac{15 \times 20}{25} = \frac{3 \times 5 \times 4 \times 5}{5 \times 5} = 3 \times 4 = 12. \]
\textbf{Solution 2:} As before, we get $AC = 25$ using Pythagoras' Theorem.
Now we can calculate the area of triangle $ABC$ in two ways: firstly, since $AB$ is perpendicular to $BC$ the area is $\frac{1}{2}b\times h = \frac{1}{2} AB \times BC = \frac{1}{2} \times 15 \times 20 = 150$.
Secondly, since $BD$ is perpendicular to $AC$ we can calculate the area as $\frac{1}{2} b\times h = \frac{1}{2} AC \times BD = \frac{1}{2} \times 25 \times BD$.
These two formulas for the area must give the same answer, so
\[ \frac{1}{2} \times 25 \times BD = 150 \implies BD = \frac{150}{\frac{1}{2} \times 25} = \frac{150 \times 2}{25} = \frac{25 \times 6 \times 2}{25} = 6 \times 2 = 12. \]


\item Quite simply, since the smaller and larger square share the same centre, the shaded area is $\frac{1}{4}$ of the difference of the two squares (since it is congruent to the regions $BFGC$, $CGHD$, and $DHEA$.
The area of the smaller square is $6^2=36$ units$^2$ and the area of the larger square is $12^2 = 144 = 144$ units$^2$.
So the area of the shaded region is $\frac{1}{4}\times (144-36) = 27$ units$^2$.

\item Let $u=\sqrt{2x}$ and let $v=\sqrt{y}$. Note that the second equation can be simplified to 
\begin{equation*}
    \sqrt{8x}+\sqrt{9y}= 2\sqrt{2x}+3\sqrt{y}.
\end{equation*}
We can then write everything in terms of $u$ and $v$. 
\begin{equation*}
    \sqrt{2x}+\sqrt{y}= u+v=13 \quad\text{and}\quad \sqrt{8x}+\sqrt{9y}=2u+3v=35.
\end{equation*}
We then solve the simultaneous equations 
\begin{equation*}
    u+v=13 \quad \text{and}\quad 2u+3v=35.
\end{equation*}
Multiplying the first equation by $2$ gives $2u+2v=26$ and subtracting from the second equation gives $v=9$. We then get that $u= 13-v=13-9=4$. We then have 
\begin{equation*}
    \sqrt{2x}=u=4 \implies x=8 \quad\text{and}\quad \sqrt{y}=v=9 \implies y=81.
\end{equation*}

\item Simplify the expression;
\begin{align*}
    2\cdot 4^{12}\cdot5^{25} &= 2\cdot(2^{2})^{12}\cdot5^{25}\\
    &=2\cdot2^{24}\cdot5^{25}\\
    &=2^{25}\cdot5^{25}\\
    &=(2\cdot5)^{25}\\
    &=10^{25}.
\end{align*}
\textbf{So there are $26$ digits in total}.

\item The prime factorization of $2024$ is $2^{3}\cdot23\cdot11$. Since $y>0$, we also have $\sqrt{y}>0$. Then 
\begin{equation*}
    0<\sqrt{x}< \sqrt{2024} \implies x<2024.
\end{equation*}
On the other hand, we have 
\begin{equation*}
    y= (\sqrt{2024}-\sqrt{x})^{2}=2024+x-2\sqrt{2024x}. 
\end{equation*}
It follows that $y$ is an integer if and only if $2024x$ is a perfect square or when $2^{3}\cdot11\cdot23x$ is a perfect square. This happens if and only if $x=2\cdot11\cdot23t^{2}$, where $t$ is an integer. Since $x$ is less than $2024$, we have 
\begin{equation*}
    2\cdot23\cdot11t^{2}<2024=2^{3}\cdot23\cdot11 \implies t^{2}<2^{2}=4.
\end{equation*}
Since $x \neq 0$, we must have that $1\leq t<2$.
Thus $t = 1$, giving $x = 506$ and
\[ y = \left(\sqrt{2024}-\sqrt{x}\right)^2 = \left(2\sqrt{506}-\sqrt{506}\right)^2 = \sqrt{506}^2 = 506. \]
Since $x$ is also equal to $506$ and $x < y$,
\textbf{there is no  solution}.

\item Tarryn can choose exactly one of each of the 3 kinds (burger, drink, dessert), in which case she has $10\times 4\times 3$ possible choices or exactly 1 of the 2 kinds (burger, drink) or (burger, dessert) or (drink, desert) in which case she has $10\times 4 + 10\times 3 + 4\times 3$ possible choices or she can choose exactly one of the 3 kinds, in which case she has $10+4+3$ possible choices. So in total, Tarryn has 
\[ 10\times 4\times 3 + 10\times 4 + 10\times 3 + 4\times 3 + 10 + 4 + 3 = 219 \] possible meals to choose from.

\item The amount of permutations of five letters is $5! = 120$.
However, NAEEM has two letters which are the same, E and E, so we have overcounted each permutation by a factor of $2$ (since we can swap the positions of E and E and get the same arrangement).
Hence the actual number of permutations here is $5!/2 = 120/2 = 60$.

\item The number of ways of choosing five stones from the $10$ available is $\binom{10}{5}$.
However, this includes possibilities where both the black and the white stone are taken, so let's count how many of these there are.
If we are choosing five stones, but two of them are white and black, then we still have to choose three stones from the remaining eight stones.
So the number of `bad' choices is $\binom{8}{3}$.
Thus the final answer is the total number of choices minus the `bad' choices, i.e.
\begin{align*}
    \binom{10}{5} -\binom{8}{3} &= \frac{10!}{5!5!} -\frac{8!}{5!3!} = \frac{10.9.8.7.6.5.4.3.2.1}{5.4.3.2.1 \times 5.4.3.2.1} -\frac{8.7.6.5.4.3.2.1}{5.4.3.2.1 \times 3.2.1} \\
    &= \frac{10.9.8.7.6}{5.4.3.2.1} -\frac{8.7.6}{3.2.1} = 2 \times 9 \times 2 \times 7 - 8 \times 7 = 252 -56 = 196.
\end{align*}

\end{enumerate}

\end{document}